\documentclass{article}
\usepackage{geometry}
\geometry{left=1.2in, right=1.2in, top=1.2in, bottom=1.2in}%change the margins here
\usepackage[utf8]{inputenc}
\usepackage{tikz}
\usetikzlibrary{cd}
\usetikzlibrary{shapes.geometric,arrows,positioning,fit,calc,}
\usepackage[english]{babel}
\usepackage[]{amsthm} %lets us use \begin{proof}
\usepackage[]{amssymb} %gives us the character \varnothing
\usepackage{mathtools}
\usepackage{amsmath}
\usepackage[shortlabels]{enumitem}
\usepackage{biblatex}
%\addbibresource{references.bib}  % The filename of your .bib file
\usepackage{csquotes}
\usepackage{float}
\usepackage[all]{xy}
\usepackage{mathrsfs}
\usepackage{multirow}
\usepackage{dsfont}
\usepackage{adjustbox}
\newcommand{\abs}[1]{\left| #1 \right|}
\newcommand{\norm}[1]{\left\| #1 \right\|}
\newcommand{\calO}{\mathcal{O}}
\newcommand{\R}{\mathbb{R}}
\newcommand{\T}{\mathbb{T}}
\newcommand{\N}{\mathbb{N}}
\newcommand{\Z}{\mathbb{Z}}
\newcommand{\Q}{\mathbb{Q}}
\newcommand{\C}{\mathbb{C}}
\newcommand{\rddots}{\reflectbox{$\ddots$}}
\newcommand{\F}{\mathbb{F}}
\newcommand{\id}{\mathrm{id}}
\newcommand{\ctd}{\Rightarrow \Leftarrow}
\newcommand{\actson}{\circlearrowright}
\newcommand{\Ss}{\mathbb{S}}
\newcommand{\B}{\mathbb{B}}
\newcommand{\fC}{\mathscr{C}}
\newcommand{\fI}{\mathscr{I}}
\newcommand{\fJ}{\mathscr{J}}
\newcommand{\fA}{\mathscr{A}}
\newcommand{\fB}{\mathscr{B}}
\newcommand{\fO}{\mathscr{O}}
\newcommand{\fF}{\mathscr{F}}
\newcommand{\fG}{\mathscr{G}}
\newcommand{\fH}{\mathscr{H}}
\newcommand{\fT}{\mathscr{T}}
\newcommand{\frakm}{\mathfrak{m}}
\newcommand{\altid}{\mathds{1}}
\newcommand{\nsubset}{\not \subset}
\newcommand\interior[1]{{#1}^{\circ}}
\newcommand{\Hh}{\mathbb{H}}
\newcommand{\D}{\mathbb{D}}
\newcommand{\Ab}{\mathbf{Ab}} %Abelian Groups
\newcommand{\Grp}{\mathbf{Grp}} %Groups
\newcommand{\Ring}{\mathbf{Ring}} %Rings
\newcommand{\CRing}{\mathbf{CRing}} %Commutative Rings
\newcommand{\Rng}{\mathbf{Rng}} %Rings without identity
\newcommand{\Set}{\mathbf{Set}} %Sets
\newcommand{\pSet}{\mathbf{Set}_{\bullet}} %Pointed Spaces
\newcommand{\Top}{\mathbf{Top}} %Topological Spaces
\newcommand{\pTop}{\mathbf{Top}_{\bullet}} %Pointed Topological Spaces
\newcommand{\Op}{\mathbf{Op}} %Open Subsets
\newcommand{\Vect}{\mathbf{Vect}} %Vector Spaces
\newcommand{\Man}{\mathbf{Man}} %Manifolds
\newcommand{\Mod}{\mathbf{Mod}} %Modules
\newcommand{\Mon}{\mathbf{Mon}} %Monoids
\newcommand{\Cat}{\mathbf{Cat}} %Small Categories
\newcommand{\Ssubset}{\mathbf{Subset}} %Subsets
\newcommand{\Com}{\mathbf{Com}} %Complexes
\DeclareMathOperator{\Haus}{\mathbf{Haus}} %Hausdorff Spaces
\DeclareMathOperator{\Comp}{\mathbf{Comp}} %Compact Spaces
\DeclareMathOperator{\Poset}{\mathbf{Poset}} %Partially Ordered Sets
\DeclareMathOperator{\Graph}{\mathbf{Graph}} %Graphs (Not Graph Theory)
\DeclareMathOperator{\Sch}{\mathbf{Sch}} %Schemes
\DeclareMathOperator{\AffSch}{\mathbf{AffSch}} %Affine Schemes
\DeclareMathOperator{\Grph}{\mathbf{Grph}} %Graphs in Graph Theory and Graph Homomorphisms
\DeclareMathOperator{\Rel}{\mathbf{Rel}} %Sets and Relations
\DeclareMathOperator{\CW}{\mathbf{CW}} %CW Complexes and Cellular Maps
\DeclareMathOperator{\PreSh}{\mathbf{PreSh}} %Presheaves
\DeclareMathOperator{\Sh}{\mathbf{Sh}} %Sheaves
\DeclareMathOperator{\catD}{\mathbf{D}} %Derived Category
\DeclareMathOperator{\TopGrp}{\mathbf{TopGrp}} %Topological Groups
\DeclareMathOperator{\Meas}{\mathbf{Meas}} %Measurable Spaces and measurable functions
\DeclareMathOperator{\Cob}{\mathbf{Cob}} %Cobordisms
\DeclareMathOperator{\LieAlg}{\mathbf{LieAlg}} %Lie Algebras
\DeclareMathOperator{\Ban}{\mathbf{Ban}} %Banach Spaces and Bounded Linear Operators
\DeclareMathOperator{\Hilb}{\mathbf{Hilb}} %Hilbert Spaces and Bounded Linear Operators
\DeclareMathOperator{\AlgC}{\mathbf{Alg_C}} %C-Algebras where C isn't necessarily commutative
\DeclareMathOperator{\Rep}{\mathbf{Rep}} %Representations
\DeclareMathOperator{\res}{res}
\DeclareMathOperator{\End}{End}
\DeclareMathOperator{\PGL}{PGL}
\DeclareMathOperator{\Aff}{Aff}
\DeclareMathOperator{\GL}{GL}
\DeclareMathOperator{\SL}{SL}
\DeclareMathOperator{\Stab}{Stab}
\DeclareMathOperator{\im}{im}
\DeclareMathOperator{\coim}{coim}
\DeclareMathOperator{\cok}{cok}
\DeclareMathOperator{\colim}{colim}
\DeclareMathOperator{\spn}{span}
\DeclareMathOperator{\Sym}{Sym}
\DeclareMathOperator{\Hom}{Hom}
\DeclareMathOperator{\Mor}{Mor}
\DeclareMathOperator{\Nat}{Nat}
\DeclareMathOperator{\Tr}{Tr}
\DeclareMathOperator{\Bd}{Bd}
\DeclareMathOperator{\Ann}{Ann}
\DeclareMathOperator{\Int}{Int}
\DeclareMathOperator{\Char}{char}
\DeclareMathOperator{\Aut}{Aut}
\DeclareMathOperator{\supp}{supp}
\DeclareMathOperator{\rank}{rank}
\DeclareMathOperator{\diag}{diag}
\DeclareMathOperator{\sign}{sign}
\DeclareMathOperator{\glue}{glue}
\DeclareMathOperator{\kerpre}{\ker_{\text{pre}}}
\DeclareMathOperator{\cokpre}{\cok_{\text{pre}}}
\DeclareMathOperator{\impre}{\im_{\text{pre}}}
\DeclareMathOperator{\sh}{sh}
\newcommand{\sqdot}{\, \raisebox{0.5ex}{\scalebox{0.2}{$\blacksquare$}} \,}
\makeatletter
\newcommand\xtwoheadrightarrow[2][]{%
  \ext@arrow 0579{\twoheadrightarrowfill@}{#1}{#2}}
\newcommand\twoheadrightarrowfill@{%
  \arrowfill@\relbar\relbar\twoheadrightarrow}
\makeatother
\let\oldemptyset\emptyset
\let\emptyset\varnothing
\newtheorem{theorem}{Theorem}[section]
\newtheorem{corollary}{Corollary}[theorem]
\newtheorem{lemma}[theorem]{Lemma}
\newtheorem*{remark}{Remark}
\newtheorem*{lemma*}{Lemma}
\renewcommand{\qedsymbol}{$\blacksquare$}
\usepackage{lipsum}                     % Dummytext
\usepackage{xargs}                      % Use more than one optional parameter in a new commands
%\usepackage[pdftex,dvipsnames]{xcolor}  % Coloured text etc.
% 
\usepackage[colorinlistoftodos,prependcaption,textsize=tiny]{todonotes}
\newcommandx{\unsure}[2][1=]{\todo[linecolor=red,backgroundcolor=red!25,bordercolor=red,#1]{#2}}
\newcommandx{\change}[2][1=]{\todo[linecolor=blue,backgroundcolor=blue!25,bordercolor=blue,#1]{#2}}
\newcommandx{\info}[2][1=]{\todo[linecolor=OliveGreen,backgroundcolor=OliveGreen!25,bordercolor=OliveGreen,#1]{#2}}
\newcommandx{\improvement}[2][1=]{\todo[linecolor=Plum,backgroundcolor=Plum!25,bordercolor=Plum,#1]{#2}}
\newcommandx{\thiswillnotshow}[2][1=]{\todo[disable,#1]{#2}}
%
\title{Jack's Exercises}
\author{Jack Westbrook}
\date\today
%This information doesn't actually show up on your document unless you use the maketitle command below

\begin{document}
\maketitle %This command prints the title based on information entered above

%Section and subsection automatically number unless you put the asterisk next to them.

%Section and subsection automatically number unless you put the asterisk next to them.
\section*{Problem 1}
\subsection*{Question}
Suppose $a$ is an integer where $\mathrm{ord}_2(a)=1$. Prove that there are no integer solutions to the equation $y^{2m}=x^{2n}+a$ for $n,m\ge 0$.
\subsection*{Answer}
\begin{proof}
    $2\mid K$ with multiplicity $1$ is equivalent to $K\equiv 2\mod 4$. We recall that squares are either $0$ or $1\mod 4$, so $x^{2n},y^{2m}\equiv 0$ or $1\mod 4$. These two facts prove the result when looking at the equation $\mod 4$.
\end{proof}

\section*{Problem 2}
\subsection*{Question}
Prove that there are no integral points on the elliptic curve $y^2=x^3-9$.
\subsection*{Answer}
\begin{proof}
    Suppose we have an integer solution pair $(x,y)$. Notice that $x$ cannot be even; if it were, then $x^3\equiv0\mod 4$; on the other hand, $y^2+9\equiv 1$ or $2\mod 4$.
    
    \vspace{0.1in}
    \noindent Now, we rewrite our equation as follows:
    \[
    (x-2)((x+1)^2+3)=y^2+1.
    \]
    Because $x$ is odd, the term $(x+1)^2+3\equiv 3\mod 4$; as such, there exists some prime $p\equiv 3\mod 4$ dividing $(x+1)^2+3$. Then we get
    \[
    y^2+1=x^3-8\equiv 0\mod p.
    \]
    But of course, this is also impossible because $-1$ is a square mod $p\ne 2$ if and only if $p\equiv 1\mod 4$.
\end{proof}
\begin{proof}
    Suppose we have an integral solution to $x^3= y^2+9$. In $\Z[i]$, we would then have $x^3=(y+3i)(y-3i).$ First, we will show that $1=(y+3i,y-3i)$. Letting $d$ be the gcd in $\Z[i]$, we have $d\mid y+3i-(y-3i)=6i$. Now the only prime that lies over $2$ in $\Z[i]$ is $1+i$, and as $3$ is inert in $\Q(i)/\Q$, we get that, because we may modify $d$ by units, $d=(1+i)^a 3^b$. If $b>0$, then $3\mid d$ implies that $0\equiv y+3i \equiv y \mod 3$. But then $x^3=y^2+9 \equiv 0 \mod 9$ implies that $3\mid x$ as well, so by replacing $x$ by $x/3$ and $y$ by $y/3$, we get solutions to the new equation $3x^3=y^2+1.$ But then
    \[
    y^2+1 = 3x^3 \equiv 0 \mod 3
    \]
    implies that $-1$ is a quadratic residue modulo 3, which is obviously false. Thus we have $d=(1+i)^a$. If $a>0$, then
    \[
    0\equiv y+3i \equiv y-3 \mod 1+i.
    \]
    Then for some $\alpha, \beta \in \Z$, we have $y-3=(\alpha +\beta i)(1+i) = \alpha-\beta+(\alpha+\beta)i.$ This implies $-\beta = \alpha$, and then that $2\alpha = y-3$, hence $y\equiv 1 \mod 2.$ Looking at the equation $x^3=y^2+9$, we then see that
    \[
    x^3=y^2+9 \equiv 0 \mod 2
    \]
    implying that $x\equiv 0 \mod 2$. Therefore $x^3\equiv 0 \mod 8$, but then
    \[
    y^2+1 \equiv y^2+9 = x^3 \equiv 0 \mod 8
    \]
    which is impossible as $-1$ is not a quadratic residue modulo 8. This proves that $d=1$. Now that $y^2+9=(y+3i)(y-3i)$ is a perfect cube, and the latter two factors are coprime, each factor must be a perfect cube. This assertion uses the fact that $\Z[i]$ is a UFD.

    Thus, for some integers $a,b$ and unit $u\in \Z[i]^*$, we have a solution to
    \[
    u(a+bi)^3 = y+3i.
    \]
    It's easy to verify that the units in $\Z[i]$ are $\{\pm 1, \pm i\}$ by looking at the norms of elements in $\Z[i]$ and recalling that an element of the ring of integers is a unit iff it has norm $1$. Expanding the above equation, we have
    \[
    y+3i = u(a^3-3ab^2+i(3a^2b-b^3)).
    \]
    Given our classification of what $u$ must be, we must either have a solution to
    \[
    \pm (3a^2b-b^3) = 3
    \]
    or 
    \[
    \pm (a^3-3ab^2) = 3.
    \]
    Let's first show that there are no solutions to the first equation. If there were, we would have $b^3 \equiv 0 \mod 3$, and thus $3\mid b.$ But then $9\mid 3a^2b-b^3$, so also $9\mid 3$ which is absurd.

    Now let's show there are not solutions to the latter equation. If there were then $3\mid a$ by considering the equation modulo 3, but then $9\mid a^3 - 3ab^2$ so $9\mid 3$ as well, again absurd.
\end{proof}
\section*{Problem 3}
\subsection*{Question}
Prove that there are no integral points on the elliptic curve $y^2=x^3-62$.
\subsection*{Answer}
\begin{proof}
    First of all, we notice that a solution to the equation $y^2=x^3-62$ if and only if there is a solution to the equation $y^2+x^3+62=0$, by replacing $x$ by $-x$. Thus it suffices to show there is no integer solution to the latter equation.
    
    Supposing $x,y$ are integers solving the equation, we can rule out $x$ being even as follows: if $x$ were even, then
    \[
    y^2-2=-(x^3+64)\equiv 0\mod 8.
    \]
    However, $2$ is not a square mod 8.

    \vspace{0.1in}
    \noindent Now we rewrite our equation as follows:
    \[
    y^2-2=-(x+4)((x-2)^2+12).
    \]
    Because $x$ is odd, $(x-2)^2\equiv 1\mod 8$, so $(x-2)^2+12\equiv -3\mod 8$. Then there exists some prime $p\equiv \pm 3\mod 8$ dividing $(x-2)^2+12$ as the only solution to $ab\equiv \pm 3\mod 8$ is $a\equiv \pm 1\mod 8$ and $b\equiv \pm3\mod 8$. But then we get that
    \[
    y^2-2=(x+4)((x-2)^2+12)\equiv 0\mod p
    \]
    which is impossible because 2 is a square mod $p$ if and only if $p\equiv \pm 1\mod 8$.
\end{proof}
\section*{Problem 4}
\subsection*{Question}
Prove there are no integer solutions to the equation $$y^2-3=x^{16}+2x^{14}+3x^{12}+4x^{10}+5+6x^2+7x^4+8x^6+9x^8.$$
\subsection*{Answer}
\begin{proof}
    We rewrite the equation as
    \[
    y^2-3=(x^8+x^6+x^4+x^2+5)(x^8+x^6+x^4+x^2+1).
    \]
    Notice now that the quadratic residues in $\Z/12\Z$ are $0,1,4$ and $9$; moreover, $x^4=x^2$ for every $x\in \Z/12\Z$. Thus, supposing a solution pair exists, 
    \[
    y^2-3\equiv (4x^2+5)(4x^2+1)\mod 12.
    \]
    The right hand side is $5 \mod 12$ when $3\mid x$, which is impossible as $8$ is not a quadratic residue in $\Z/12\Z$. If $3\nmid x$, $x^2\equiv 1$ or $4\mod 12$; in either case, $4x^2+1\equiv 5\mod 12$.

    \vspace{0.1in}
    \noindent Because every prime greater than 2 must be congruent to either $\pm 1$ or $\pm 5\mod 12$, we conclude that there is some prime $p\equiv \pm 5\mod 12$ dividing $4x^2+1$ (because $4x^2+1$ is odd and has a prime factorization), and thus also $y^2-3$. We then have
    \[
    y^2-3\equiv 0\mod p
    \]
    which is impossible because $3$ is a quadratic residue in $\Z/p\Z$ if and only if $p=2$, $p=3$, or $p\equiv \pm 1 \mod 12$. To prove this, we first easily notice that $3$ is a quadratic residue modulo $2$ and $3$. For $p>3$, we compute that by quadratic reciprocity, if $p\equiv 1 \mod 4$ we have $(\frac{3}{p})=(\frac{p}{3})$, and if $p\equiv 3\mod 4$, then $(\frac{3}{p})=-(\frac{p}{3})$. It's also easy to see that $$(\frac{p}{3})=\begin{cases}
        1, & \text{ if } p\equiv 1 \mod 3\\
        -1, & \text{ if } p \equiv 2 \mod 3.
    \end{cases}$$
    Now we can easily see that $(\frac{3}{p})=1$ precisely when $p\equiv 1 \mod 4$ and $p\equiv 1 \mod 3$, or when $p\equiv 3 \mod 4$ and $p\equiv 2\mod 3$, or equivalently when $p\equiv 1 \mod 12$ or $p\equiv 11 \mod 12$.
\end{proof}
\section*{Problem 5}
\subsection*{Question}
Prove that there are no integer solutions to $2y^2=2x^4+3x^2+1$.
\subsection*{Answer}
\begin{proof}
    We first notice that a solution pair exists to our given equation if and only if a solution pair exists for the equation $y^2=16x^4+24x^2+8$. This is because if $(x,y)$ satisfies our original equation, then $(x,4y)$ satisfies the new equation as
    \[
    (4y)^2=16y^2=8(2x^4+3x^2+1)=16x^4+24x^2+8,
    \]
    and conversely if $(x,y)$ satisfies the new equation, then $y\equiv 0 \mod 4$ since $y^2 \equiv 0 \mod 8$, hence $\frac{y}{4}\in \Z$, and $(x,\frac{y}{4})$ is a solution pair to the original equation because
    \[
    2 (\frac{y}{4})^2 = \frac{1}{8}(16x^4+24x^2+8)=2x^4+3x^2+1.
    \]
    We will now consider the equation $y^2=16x^4+24x^2+8$, and rewrite $y^2=16x^4+24x^2+8$ as $y^2-3=16x^4+24x^2+5 = (4x^2+1)(4x^2+5)$. Notice now that the quadratic residues in $\Z/12\Z$ are $0,1,4$ and $9$. The right hand side is $5 \mod 12$ when $3\mid x$, which is impossible as $8$ is not a quadratic residue in $\Z/12\Z$. If $3\nmid x$, $x^2\equiv 1$ or $4\mod 12$; in either case, $4x^2+1\equiv 5\mod 12$.

    \vspace{0.1in}
    \noindent Because every prime greater than 2 must be congruent to either $\pm 1$ or $\pm 5\mod 12$, we conclude that there is some prime $p\equiv \pm 5\mod 12$ dividing $4x^2+1$ (because $4x^2+1$ is odd and has a prime factorization), and thus also $y^2-3$. We then have
    \[
    y^2-3\equiv 0\mod p
    \]
    which is impossible because $3$ is a quadratic residue in $\Z/p\Z$ if and only if $p=2$, $p=3$, or $p\equiv \pm 1 \mod 12$. To prove this, we first easily notice that $3$ is a quadratic residue modulo $2$ and $3$. For $p>3$, we compute that by quadratic reciprocity, if $p\equiv 1 \mod 4$ we have $(\frac{3}{p})=(\frac{p}{3})$, and if $p\equiv 3\mod 4$, then $(\frac{3}{p})=-(\frac{p}{3})$. It's also easy to see that $$(\frac{p}{3})=\begin{cases}
        1, & \text{ if } p\equiv 1 \mod 3\\
        -1, & \text{ if } p \equiv 2 \mod 3.
    \end{cases}$$
    Now we can easily see that $(\frac{3}{p})=1$ precisely when $p\equiv 1 \mod 4$ and $p\equiv 1 \mod 3$, or when $p\equiv 3 \mod 4$ and $p\equiv 2\mod 3$, or equivalently when $p\equiv 1 \mod 12$ or $p\equiv 11 \mod 12$.
\end{proof}
\begin{proof}
    Using the same rearrangement, we can prove that there are no integer solutions to $2y^2=2x^4+3x^2+1$ by only modular arithmetic. First, we notice that $x\equiv 1 \mod 2$ by taking the equation $\mod 2$. Therefore $x^2\equiv 1 \mod 4$, so we get $2y^2 \equiv 2 \mod 4$. This implies that $y\equiv 1\mod 2$ as well. Considering our equation modulo $3$, we have
    \[
    2y^2\equiv 2x^2+1 \mod 3.
    \]
    Then $x\equiv \pm 1 \mod 3$, implying $y\equiv 0 \mod 3$. Now we consider our equation modulo $5$. If $x\equiv 0 \mod 5$, then 
    \[
    2y^2 \equiv 1 \mod 5
    \]
    and as $3=2^{-1} \mod 5$, we would have $y^2\equiv 3 \mod 5$ is impossible. Thus $x\not \equiv 0 \mod 5$, hence $x^4\equiv 1 \mod 5$ and then
    \[
    2y^2 \equiv 3x^2+3 \mod 5.
    \]
    Multiplying each side by $3$, we have $y^2 \equiv -(x^2+1) \mod 5$. But as $x^2 \equiv \pm 1 \mod 5$, we notice there is no solution if $x^2\equiv 1 \mod 5$, and thus $x^2 \equiv -1 \mod 5$. Thus $x\equiv \pm 2 \mod 5$ and $y\equiv 0 \mod 5$. Thus $y\equiv 15 \mod 30$ and $x$ can only be congruent to one of $\pm 7, \pm 13 \mod 30$. However, we then check that $2x^4$
\end{proof}
\section*{Problem 6}
\subsection*{Question}
Show that there are no integer solutions to the equation $y^2=4x^4+9x^2+5$.
\subsection*{Answer}
\begin{proof}
    We rewrite $y^2=4x^4+9x^2+5$ as $y^2-3=4x^4+9x^2+2 = (4x^2+1)(x^2+2)$. Notice that the quadratic residues in $\Z/12\Z$ are $0,1,4$ and $9$. The right hand side of the given equation is $2, 5 \text{ or } 6 \mod 12$ if $x^2 \not \equiv 4 \mod 12$, which is impossible as neither are quadratic residues. Thus $4x^2+1 \equiv 5 \mod 12$.

    \vspace{0.1in}
    \noindent Because every prime greater than 2 is congruent to either $\pm 1$ or $\pm 5\mod 12$, we conclude that there is some prime $p\equiv \pm 5\mod 12$ dividing $4x^2+1$ ($4x^2+1$ is odd and thus has prime divisors strictly greater than 2), hence also $y^2-3$. We then have
    \[
    y^2-3\equiv 0\mod p
    \]
    which is impossible because $3$ is a quadratic residue in $\Z/p\Z$ if and only if $p=2$, $p=3$, or $p\equiv \pm 1 \mod 12$. To prove this, we first easily notice that $3$ is a quadratic residue modulo $2$ and $3$. For $p>3$, we compute that by quadratic reciprocity, if $p\equiv 1 \mod 4$ we have $(\frac{3}{p})=(\frac{p}{3})$, and if $p\equiv 3\mod 4$, then $(\frac{3}{p})=-(\frac{p}{3})$. It's also easy to see that $$(\frac{p}{3})=\begin{cases}
        1, & \text{ if } p\equiv 1 \mod 3\\
        -1, & \text{ if } p \equiv 2 \mod 3.
    \end{cases}$$
    Now we can easily see that $(\frac{3}{p})=1$ precisely when $p\equiv 1 \mod 4$ and $p\equiv 1 \mod 3$, or when $p\equiv 3 \mod 4$ and $p\equiv 2\mod 3$, or equivalently when $p\equiv 1 \mod 12$ or $p\equiv 11 \mod 12$.
\end{proof}
\section*{Problem 7}
\subsection*{Question}
Find the greatest positive integer $n$ such that $p$ is a fourth root of unity in $\Z/n\Z$ for every prime $p\ge 11$.
\subsection*{Answer}
\begin{proof}
    We claim $n=240$ is the solution. First, we will show that $n=240$ works by proving that for every $p\ge 11$, $p^4-1\equiv 0 \mod 240$. Notice that $p^4-1=(p^2+1)(p-1)(p+1)$, where for each prime $p>2$, each factor is even. If $p\equiv 1\mod 4$ we have $p-1\equiv 0\mod 4$ and the other factors are even implies $p^4-1$ is equivalent to $0\mod 16$, and if $p\equiv 3\mod 4$ then $p+1\equiv 0\mod 4$ and the other factors even imply the result is divisible by $16$ as well. Separately, because $p>3$ implies $p\not \equiv 0\mod 3$, we have $p^2\equiv 1\mod 3$, so indeed $3\mid p^4-1$. Lastly, since $x^4\equiv 1 \mod 5$ for every integer $x$ indivisible by $5$, we automatically get $p^4-1 \equiv 0 \mod 5$ since $p>5$. Now because $16, 3,$ and $5$ are pairwise coprime and $p^4-1$ is divisible by each of them, we see $p^4-1\equiv 0 \mod  240$.
    
    \vspace{0.1in}
    For the reverse direction, suppose $p^4-1$ is divisible by $n$ for every $p\ge 11$. Let $n=\prod p_i^{\alpha_i}$ be its prime factorization. Then $p^4-1$ is divisible by $n$ if and only if it is divisible by $p_i^{\alpha_i}$ for each $i$ by the Chinese remainder theorem. We have then for any $p\ge 11$ and any $i$ that
    \[
    p_i^{\alpha_i}\mid p^4-1=(p^2+1)(p-1)(p+1)
    \]
    only if $p_i$ divides at least one of $p^2+1,p-1,$ or $p+1$ for each $i$.
    If any $p_i\ge 11$, then we let $p=p_i$ and arrive at a contradiction because $p_i$ does not divide $p_i^2+1$ or $p_i-1$ or $p_i+1$. Thus $n=2^{\alpha_1} 3^{\alpha_2} 5^{\alpha_3} 7^{\alpha_4}$. We have by assumption that $2^{\alpha_1} 3^{\alpha_2} 5^{\alpha_3} 7^{\alpha_4}\mid (11^2+1)(11-1)(11+1)=122\cdot 10\cdot 12=2^4\cdot 3\cdot 5\cdot 61$ so indeed the maximum values for $\alpha_1$, $\alpha_2$ and $\alpha_3$ are $4$, $1$ and $1$ respectively while $\alpha_4$ must be $0$. Then $n \le 2^4 \cdot 3 \cdot 5 = 240$, giving the result.
\end{proof}
\section*{Problem 8}
\subsection*{Question}
Show that the only integral points on the elliptic curve $y^2=x^3-11$ are $(3, \pm 4)$ and $(15, \pm 58).$
\begin{proof}
    Suppose $x,y\in \Z$ are such that $y^2=x^3-11.$ First, we recall that $\Z[\omega]=\calO_{\Q(\sqrt{-11})}$ has class number $1$ where $\omega = \frac{1+\sqrt{-11}}{2}$, i.e., is a PID. Then in $\Z[\omega]$, we have
    \[
    (y+\sqrt{-11})(y-\sqrt{-11}) = x^3.
    \]
    For ease of notation, we let $z=y+\sqrt{-11}$, $d=\gcd(z, \bar z)$, and $\alpha = z/d\in \Z[\omega].$
    We observe that $d\mid z-\bar z = 2\sqrt{-11}$, and that
    \[
    N(2)=4
    \]
    and
    \[
    N(\sqrt{-11})=11
    \]
    So $N(\sqrt{-11})$ prime implies $\sqrt{-11}$ is irreducible. To show $2$ is irreducible, it suffices to show there is no element in $\Z[\omega]$ with norm $2.$ Indeed, for $\zeta = a+b\omega$, we compute
    \[
    N(\zeta)=\zeta \bar \zeta = (a+b\omega)(a+b\bar \omega)=a^2+ab+3b^2
    \]
    which cannot equal $2$, because a solution would enforce
    \[
    a^2+ab+b^2 \equiv 0 \mod 2
    \]
    which implies that $a\equiv b \equiv 0 \mod 2.$ But then $4$ divides the left hand side, while $4$ does not divide $2$ obviously.

    Therefore $d=1$ or $2$ or $\sqrt{-11}$ or $2\sqrt{-11}.$ This shows $\bar d = \pm d$. Therefore $\gcd(\alpha, \bar \alpha)=1$ since $\bar \alpha = \bar z/\bar d = \pm \bar z/d$. Since
    \[
    x^3 = z \bar z = d^2 \alpha \bar \alpha
    \]
    it follows that for any irreducible $\pi \in \Z[\omega]$,
    \[
    2\nu_\pi(d)+\nu_\pi(\alpha)+\nu_\pi(\bar \alpha) \equiv 0 \mod 3
    \]
    and also that at most one of $\nu_\pi(\alpha), \nu_\pi(\bar \alpha)$ is nonzero.

    If $d=1$, then $\nu_\pi(d)=0$ for all $\pi$ implies $\nu_\pi(\alpha)\equiv 0 \mod 3$ for all $\pi$, hence $$z=d\alpha = \alpha = \zeta^3$$ for some $\zeta \in \Z[\omega].$

    If $d=2$, then 
    \[
    \nu_\pi(d)=\begin{cases}
        1, & \text{ if } \pi = 2\\
        0, &\text{ otherwise}
    \end{cases}
    \]
    so
    \[
    \nu_\pi(\alpha)+\nu_\pi(\bar \alpha) \equiv \begin{cases}
        1, & \text{ if } \pi = 2\\
        0, & \text{ otherwise}
    \end{cases} \mod 3.
    \]
    However, $2\mid \alpha$ iff $2\mid \bar \alpha$, which forces $\nu_2(\alpha)=\nu_2(\bar \alpha)=0$. This contradicts the above equation for $\pi=2$, so $d\ne 2.$

    If $d=\sqrt{-11}$ or $d=2\sqrt{-11}$, a very similar proof yields a contradiction, so we conclude $d=1$ and $z=\zeta^3$ for some $\zeta \in \Z[\omega].$ We have
    \[
    2\sqrt{-11}=z-\bar z =\zeta^3-\bar \zeta^3 = (\zeta-\bar \zeta)(\zeta^2+\zeta \bar \zeta+\bar \zeta^2).
    \]
    Letting $\zeta = a+b\omega$ with $a,b\in \Z$, we compute
    \begin{align*}
        &\zeta-\bar \zeta = b \sqrt{-11}\\
        &\zeta^2= a^2-3b^2+(b^2+2ab)\omega\\
        &\zeta \bar \zeta = a^2+3b^2+ab\\
        &\bar \zeta^2 = a^2-3b^2+(b^2+2ab)\bar \omega\\
        & \zeta^2+\zeta \bar \zeta + \bar \zeta^2 = 3a^2-2b^2+3ab\\
        &2\sqrt{-11} = b\sqrt{-11}(3a^2+3ab-2b^2)
    \end{align*}
    which by the unique factorization gives the integral equation
    \[
    2=b(3a^2+3ab-2b^2)
    \]
    which leaves four possibilities for $b$: $\pm 1$ or $\pm 2.$ If $b=2$, then
    \[
    1=3a^2+6a-8 \Rightarrow a^2+2a-3=0 \Rightarrow a = -3 \text{ or } 1.
    \]

    If $b=-2$, then
    \[
    -1=3a^2-6a-8 \Rightarrow 3a^2-6a-7 = 0
    \]
    has no integer solutions because we would get $0=3a^2-6a-7 \equiv -7 \mod 3$ is impossible.

    If $b=1$, then
    \[
    2=3a^2+3a-2\Rightarrow 3a^2+3a-4=0
    \]
    also has no integer solutions since we would get $-4 \equiv 0 \mod 3.$

    Lastly, if $b=-1$, then
    \[
    -2=3a^2-3a-2\Rightarrow a(a-1)=0 \Rightarrow a = 0 \text{ or } 1.
    \]

    Thus the only possible values of $\zeta$ are
    \[
    \zeta = 1+2\omega \text{ or } -3+2\omega \text{ or } -\omega \text{ or } 1-\omega.
    \]
    Then
    \[
    y+\sqrt{-11}=z=\zeta^3 = -58+\sqrt{-11} \text{ or } 58+\sqrt{-11} \text{ or } 4 +\sqrt{-11} \text{ or } -4+\sqrt{-11}.
    \]
    Thus $y=\pm 4, \pm 58$ are the only possible values, and correspondingly we get $x=3,15$.
\end{proof}
\section*{Problem 9}
\subsection*{Question}

Show that $f(X)=X^6-108\in \Q[X]$ is irreducible.
\begin{proof}
    By Gauss' Lemma, this polynomial is irreducible over $\Q$ iff it's irreducible over $\Z$, since it's primitive. Thus it suffices to show its irreducible over $\F_p$ for some prime $p$, since factorization over $\Z$ gives factorization in $\F_p.$ For $p=7$, we get $\bar f(X)= X^6-3\in \F_7[X].$ We recall that $\F_q^\times$ is cyclic for every prime power $q.$ Thus $\bar f$ has no roots in $\F_7$ since $x^6=1$ for all $x\in {\F_7}^\times.$ If $\bar f$ had a quadratic factor in $\F_7$, then by modding out this quadratic factor from $\F_7[X]$, we would get a root of $\bar f$ in $\F_{7^2}.$ Thus let $x\in \F_{7^2}^\times$ be such that $x^6=3.$ But since $\F_{7^2}^\times$ is cyclic of order $48$, it follows that the sixth powers form a subgroup of order $8$, so then $1=3^8=3^2=2$, a contradiction.

    Then the only remaining possibility is that $\bar f$ has a cubic factor. As before, this implies that $\bar f$ has a root in $\F_{7^3}^\times.$ Since $\F_{7^3}^\times$ is cyclic of order $342$, the sixth powers form a subgroup of order $57.$ But $3^{57}=(3^6)^9 \cdot 3^3=1^9 \cdot 27=3$ which again is a contradiction.
\end{proof}
\end{document}